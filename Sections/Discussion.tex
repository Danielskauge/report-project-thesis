\section{Discussion}

\subsection{Model Limitations and Uncertainties}

\subsubsection{Symmetric Jump Model Limitations}
The symmetric jumping configuration approximates asymmetric jumping performance. For asymmetric legs, achieving vertical jumps requires careful paw placement relative to hip joints, varying with link lengths. While symmetric models place paws directly under hips for L1$=$L2 and L1$>$L2, this produces angled rather than vertical jumps in asymmetric cases.

Experiments show vertical jumps are possible with asymmetric legs by adjusting hip-to-paw vector angle. However, the symmetric model only partially captures this through a constant angle offset when $L2 > L1$, failing to produce vertical jumps across all link lengths.

This introduces systematic biases:
\begin{itemize}
    \item When \(L_1 \neq L_2\): Underestimates jump height since asymmetric vertical jumps require initial poses that angle L2 and L1 more horizontally, better converting angular to vertical velocity with less slipping.
    \item When \(L_1 = L_2\): Overestimates jump height since achieving vertical jumps requires L2 to be less horizontal, where they are fully horizontal in the symmetric case \ref{fig:link_length_optimization:equal_len_pose}.
\end{itemize}

These biases explain the sharp performance peak at L2 = L1 in the vertical jump grid search results for both Earth (figure \ref{fig:results:grid_search_earth}) and Mars (figure \ref{fig:results:grid_search_mars}) gravity.

\subsubsection{Unmodeled Dynamics}
While simulation results are promising, the lack of hardware testing introduces uncertainties about actual jumping performance. The 
simulated optimal link lengths may differ from real-world optima due to several unmodelled dynamics and approximations. There are 
unmodelled frictions like those between the springs and the legs. There are mass distribution differences from approximating legs and 
torso as uniform density rectangular prisms, and there are likely differences in the real and simulated masses of the robot.

\subsubsection{Motor Loading Uncertainty}
A critical uncertainty is the knee motors' ability to maintain stall torque while loading springs. As discussed in section \ref{sec:bldc_theory}, prolonged stall torque risks motor overheating. The time required to fully load springs at stall torque remains unknown, making it unclear whether motors can safely achieve maximum spring compression. Our 12\% safety margin for stall torque falls below the recommended 20\% discussed in section \ref{sec:bldc_theory}, increasing risk that motors cannot load springs maximally.

\subsection{Design Limitations}

The optimal link lengths found for both Earth and Mars are too long for the current robot design. To avoid leg collision with bent knees, the longer lengths would require either elongating the robot body, or translating the hip joints outward. Body elongation would increase overall weight, while outward hip translation raises collision risks during aerial stabilization. Additionally, longer legs increase the inertia that motors must overcome during aerial maneuvers, potentially slowing aerial stabilization response.

Given these constraints, we must opt for shorter link lengths without sacrificing significant performance. 

A potential fix would be to incorporate body length into the optimization process. Additionally, activating hip motors during jumps may yield shorter optimal link lengths.

\subsubsection{Landing Challenges}
The knee springs' equilibrium position at $\theta_2=0$ (straight leg) complicates landing. Achieving crouched landing poses requires knee motors to work against spring force, increasing response time. Further testing is needed to quantify motor response times across knee angles.

\subsubsection{Hip Abduction/Adduction Design}
As seen in figure \ref{fig:assembly_CAD}, the current leg design has a paw position that isn't located directly below the axis of rotation of the hip abduction/adduction motor. In other words, the hip motors will have to produce a powerful torque not to be rotated by vertical end-effector forces. As the specialization project has primarily ignored abduction/adduction dynamics, adjusting the shank design to avoid this issue is considered an avenue of future work.

\subsection{Post-takeoff rotational velocity}
The common knee bending direction shifts the center of mass backward, causing the rear legs to support more weight while the front legs 
experience less resistance to acceleration. This imbalance generates backward rotation at takeoff, making aerial stabilization more 
challenging. Two potential solutions are to increase the spring stiffness of the rear knee springs or activate hip motors during jumps with greater torque applied to the back legs.


\subsubsection{Limited Jump Control}
Current design lacks motor actuation during jumps, preventing feedback control from compensating for parameter variations like spring stiffness. This could widen the sim-to-real gap. Adding hip motor actuation during jumps would enable feedback control - a focus for future masters thesis work.


