\section{Discussion}

\subsection{Link Length Optimization Trade-offs}
The optimal link lengths found for both Earth and Mars gravity present practical challenges for the mechanical design. The longer lengths would require either elongating the robot body to prevent leg collisions, or translating the hip joints outward. Body elongation would increase overall weight, while outward hip translation raises collision risks during aerial stabilization. Additionally, longer legs increase the inertia that motors must overcome during aerial maneuvers, potentially slowing stabilization response.

Given these constraints, we opted for shorter link lengths that balance jumping performance with mechanical practicality. Future work could incorporate body dimensions into the optimization process. Additionally, activating hip motors during jumps may yield different optima that better align with design constraints.

\subsection{Limitations of the Symmetric Jump Model}
The symmetric jumping configuration has key limitations in approximating asymmetric jumping performance. For asymmetric legs, achieving a specific jump angle requires careful paw placement relative to the hip joints - a relationship that varies with link lengths. While the symmetric model places paws directly under hips, this same placement in the asymmetric case produces angled rather than vertical jumps.

Experiments show vertical jumps are possible with asymmetric legs by adjusting the hip-to-paw vector angle. However, the symmetric model only partially captures this through a constant angle offset when $L2 > L1$, which fails to produce vertical jumps across all link lengths.

This approach introduces systematic biases:
\begin{itemize}
    \item When \(L_1 \neq L_2\): Underestimates jump height since asymmetric vertical jumps require paw positions that angle L2 and L1 more horizontally, better converting angular to vertical velocity with less slipping
    \item When \(L_1 = L_2\): Overestimates jump height since achieving vertical jumps requires L2 to be less horizontal, reducing velocity conversion efficiency
\end{itemize}

These biases help explain the sharp performance peak at L2 = L1 in the grid search results (figure \ref{fig:link_length_optimization:grid_search_results}) for both Earth and Mars gravity.

\subsection{Control and Dynamic Challenges}
Because the current plan is to turn the knee motors off during jumps, because the torsional springs' power is so much higher than the motors peak power, and it is unknown how the motors will handle that, the current design lacks knee feedback capability during jumps. This could limit the future RL controller's ability to compensate for parameter variations like spring stiffness. This could potentially widen the sim-to-real gap. As well as limit the robots ability to jump accurately. 

When knees bend in the same direction, the shifted center of mass causes uneven leg loading. The back legs bear more weight while the front legs accelerate faster, resulting in backward rotation that complicates aerial stabilization. These challenges could be addressed by using stronger rear springs/motors or by having the hip motors on during the jump with more power provided to the back legs.

\subsection{Design Considerations for Landing}
The knee springs' equilibrium position at $\theta_2=0$ (straight leg) may complicate landing. To achieve a crouched landing pose, knee motors must work against spring force, increasing response time. Further testing is needed to quantify motor response times for various knee angles.

\subsection{Lack of Hardware Tests}
While simulation results are promising, the absence of hardware testing introduces several uncertainties. The simulated optimal link lengths may differ from real-world optima due to unmodeled dynamics and approximations. These include but are not limited to: 
\begin{itemize}
    \item Unmodeled frictions, for example friction between springs and motor housing
    \item The transversal motors' ability to withstand torque around their rotation axis during jumps
    \item The actual mass distribution likely differs from current models, which approximate legs and torso as uniform density rectangular prisms
    \item The torso mass and inertia properties are approximated as described in section \ref{sec:mass_inertia_properties}, which may not be accurate
    \item Discrepancies in masses such as the leg mass. As described in section \ref{sec:robot_hardware}, the simulated and actual leg masses differ, but the difference in mass is considered negligible compared to motor and body masses
    \item Simplified viscous friction models for motors
\end{itemize}

These approximations, combined with simplified torso mass and inertia properties detailed in section \ref{sec:mass_inertia_properties}, may lead to meaningful discrepancies between simulation and reality.

\subsection{Motor Loading Uncertainty}
A critical uncertainty is the knee motors' ability to maintain stall torque while loading the springs. As discussed in section \ref{fig:bldc_torque_speed}, prolonged stall torque risks motor overheating. The time required to fully load springs at stall torque remains unknown, making it unclear whether motors can safely achieve maximum spring compression. This uncertainty could significantly impact the robot's jumping performance and motor longevity.

As detailed in section \ref{sec:motor_modeling}, the recommended safety margin for stall torque is about 20\%. Our margin is smaller at 12\%, which increases the risk that the motors will not be able to load the springs maximally.

\subsection{Lack of feedback control during jumps}
The current design lacks motor actuation and thus feedback control during jumps, which stops the future RL controller's ability to compensate for parameter variations like spring stiffness. This could potentially widen the sim-to-real gap. Introducing hip motor actuation during jumps would allow feedback control, which is why it will be investigated in the upcoming masters thesis.

\subsection{Hip Abduction/Adduction Vector Arm}
\label{sec:hip_abduction_adduction_vector_arm}

As can be seen in figure \ref{fig:assembly_CAD}, there is a considerable distance in the $z$ direction from the end of the shank, ie. where the paw will touch the ground, to the axis of rotation of the hip abduction/adduction motor. This distance will cause a significant and undesired load on the hip abduction/adduction motor. As the specialization project has primarily ignored abduction/adduction dynamics, adjusting the shank design to avoid this issue is considered an avenue of future work.

