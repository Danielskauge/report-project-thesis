\section{Motor Friction Estimation}
\label{sec:motor_friction_estimation}

This section details the estimation of parameters of the motor friction models for the hip and knee motors. The motor act as the pivot point for a pendulum, which consists of an aluminum rod attached to the motor shaft with a ballast mass attached. The pendulum is released from a horizontal position and the angular velocity and acceleration are measured. These measurements allow us to estimate the motor friction coefficients using linear regression. The experimental setup for the big motor is shown in figure \ref{fig:results:motor_friction_estimation:pendulum_setup}.

We chose a linear friction model as described in section \ref{sec:motor_friction_models}, consisting of only viscous friction, as it is a good fit for the data.

Pendulum angles were sampled by manually annotating the pendulum angle in the video frames at 60hz, using the program Tracker \cite{tracker}. 

The angular velocity \( \dot{\theta} \) and acceleration \( \ddot{\theta} \) are computed using centered finite differences:

\[
\dot{\theta}_i = \frac{\theta_{i+1} - \theta_{i-1}}{2\Delta t}
\]

\[
\ddot{\theta}_i = \frac{\theta_{i+1} - 2\theta_i + \theta_{i-1}}{(\Delta t)^2}
\]

where \(\Delta t\) is the time step between measurements.





\begin{figure}[h]
    \centering
    \includegraphics[width=0.6\textwidth]{Images/friction_estimation/big_motor.jpg}
    \caption{Experimental setup for motor friction estimation. An aluminum rod with ballast mass attached acts as a pendulum, with the motor shaft as the pivot point.}
    \label{fig:results:motor_friction_estimation:pendulum_setup}
\end{figure}

\subsection{Pendulum Modeling}
The pendulum used in the motor friction tests consists of an aluminum rod of length \( l_{\text{arm}} = 0.21 \) meters and a ballast mass \( m_{\text{ballast}} = 0.301 \) kg attached at a distance \( d_{\text{ballast}} = 0.20 \) meters from the pivot. The total mass of the arm is \( m_{\text{arm}} = 0.034 \) kg. The pendulum is modeled as a rigid body rotating about the motor shaft with a moment of inertia \( I \) given by:



\[
I = \frac{1}{3} m_{\text{arm}} l_{\text{arm}}^2 + m_{\text{ballast}} r^2
\]

The equation of motion for the pendulum, considering only viscous friction, is:

\[
I \ddot{\theta} + b \dot{\theta} + (m_{\text{arm}} \frac{l_{\text{arm}}}{2} + m_{\text{ballast}} r) g \sin(\theta) = 0
\]

where:
\begin{itemize}
    \item \( \theta \) is the angular displacement (positive counterclockwise, zero at vertical down position)
    \item \( \dot{\theta} \) and \( \ddot{\theta} \) are the angular velocity and acceleration, respectively
    \item \( b \) is the viscous damping coefficient
    \item \( g = 9.81 \, \text{m/s}^2 \) is the acceleration due to gravity
\end{itemize}

\subsection{Linear Regression Derivation}
Rearranging the equation for linear regression purposes:

\[
    I \ddot{\theta} + (m_{\text{arm}} \frac{l_{\text{arm}}}{2} + m_{\text{ballast}} d_{\text{ballast}}) g \sin(\theta) = -b \dot{\theta}
\]

This can be expressed in the form:

\[
Y = X \beta
\]

where:
\begin{itemize}
    \item \( Y = -I \ddot{\theta} - (m_{\text{arm}} \frac{l_{\text{arm}}}{2} + m_{\text{ballast}} d_{\text{ballast}}) g \sin(\theta) \),
    \item \( X = \dot{\theta} \),
    \item \( \beta = b \).
\end{itemize}

The linear least squares solution for \( \beta \) is given by:

\[
\beta = (X^T X)^{-1} X^T Y
\]

This yields the viscous damping coefficient \( b \).

Results are detailed in section\ref{sec:results:motor_friction_estimation}.