\section{Theory}

\subsection{Reinforcement Learning}
    \subsubsection{The Reinforcement Learning Problem Setting}
    \subsubsection{Value-Based and Policy-Based Methods}
    \subsubsection{Policy Gradient Methods}

\subsection{Artificial Neural Networks}

\subsection{Electric Motor Dynamics}
    \subsubsection{Motor Modeling}
    \subsubsection{Torque Speed Models}

\subsection{Spring-Damper Systems}

\subsection{Kinematics, Jacobians, and Virtual Work}
    \subsubsection{Robot Kinematics}
    \label{sec:robot_kinematics}
Consider a robotic link arm existing in $\mathbb{R}^2$ consisting of $n$ links, each with a length $l_i$ and a joint angle $\theta_i$. The position of the end-effector is given by the vector $\mathbf{x} = [x, y]^T$, where $x$ and $y$ are the coordinates of the end-effector in the global coordinate system. Using simple trigonometry, the position of the end-effector can be expressed as a function of the joint angles and link lengths as seen in equation \ref{eq:robot_kinematics}.

\begin{equation}
    \label{eq:robot_kinematics}
    \mathbf{x} = \begin{bmatrix}
        x \\
        y
    \end{bmatrix} = \begin{bmatrix}
        \sum_{i=1}^{n} l_i \cos(\theta_i) \\
        \sum_{i=1}^{n} l_i \sin(\theta_i)
    \end{bmatrix}
\end{equation}

% This simple relation is illustrated in figure \ref{fig:robot_kinematics}. 

    \subsubsection{Jacobian Matrix}

As described in section \ref{sec:robot_kinematics}, the position of the end-effector can be expressed as a function of the joint angles and link lengths. In this section we define the Jacobian, as described in equation \ref{eq:jacobian}, which relates infitesimal changes in the joint angles to the resulting change in the end-effector position.

Now I am testing how quick this is. Testing again. So much testing. 

\begin{equation}
    \label{eq:jacobian}
    \mathbf{J} = \begin{bmatrix}
        \frac{\partial x}{\partial \theta_1} & \frac{\partial x}{\partial \theta_2} & \cdots & \frac{\partial x}{\partial \theta_n} \\
        \frac{\partial y}{\partial \theta_1} & \frac{\partial y}{\partial \theta_2} & \cdots & \frac{\partial y}{\partial \theta_n}
    \end{bmatrix}
\end{equation}


    \subsubsection{Virtual Work and Force/Torque Mapping}

    Example citation \cite{schulman_proximal_2017}

\subsection{Dynamical Systems and Contact Dynamics}
    \subsubsection{Rigid-Body Dynamics}
    \subsubsection{Contact Modeling and Impact Dynamics}
    \subsubsection{Friction Models}
