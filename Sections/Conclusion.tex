\section{Conclusion}
\label{sec:conclusion}

This project developed a small, lightweight and low-cost jumping quadruped that combines parallel torsional springs with BLDC motors. It is intended to be used as a platform for training DRL policies for jumping, arial stabilization, and landing.

Simscape simulations showed motor-only actuation produced insufficient jump heights, leading to the addition of torsional springs. Grid search optimization revealed that equal-length links maximize jump height, though the optimal lengths were too long for a compact robot design.

Motor friction estimation through pendulum tests improved simulation accuracy, but hardware testing remains incomplete. We have not tested if motors can maintain stall torque during spring loading without overheating. Unmodeled dynamics like spring-leg interface friction and mass distribution differences will affect real performance. The symmetric jumping model differs from realistic asymmetric leg configurations, and simplified paw placement affects jump trajectory predictions.

The current design has several limitations. Without motor actuation during jumps, potential energy remains untapped. Lack of feedback control during jumps prevents adjustment for unmodeled dynamics differences between simulation and real hardware. Spring equilibrium at straight leg slows landing response. Hip abduction/adduction motor placement creates undesired loading, and the common knee bending direction generates backward rotation at takeoff.

Despite these limitations, simulations indicate the design can achieve sufficient jump heights for demonstrating DRL controllers in both Earth and Mars gravity, setting the foundation for future hardware validation and control development in the upcoming master's thesis.
