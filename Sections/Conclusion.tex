\section{Conclusion}
\label{sec:conclusion}

This project developed a concept design for a small and lightweight jumping quadruped robot to be used as a platform for training DRL controllers for jumping, arial stabilization, and landing. 

Through motor selection and Simulink Simscape simulations, motor-only actuation proved inadequate for achieving the desired jump performance. Consequently, integrating torsional springs was essential to store and release energy effectively during jumps.

Grid search optimization identified optimal link lengths that to maximize effective jump height. Optimal link lengths were found to be too long for practical robot construction, so link lengths were chosen according to a trade-off between jump height and robot size. System identification refined motor friction models, enhancing simulation accuracy.

Limitations include the symmetric jumping model's simplifications yielding potentially different optimal link lengths than a more realistic asymmetric jumping model. We have not yet performed hardware tests for jumping, thus we do not know how realistic the simulation results are, given mass, inertia, and friction approximations and unmodeled dynamics. We have not tested if the knee motors are able to give stall torque for long enough to load the springs, without overheating or damaging the motor, which is it assumed they can do in simulation.

Despite these limitations, the project demonstrates that the robot design is capable of sufficient jumping performance for demonstrating DRL controllers for jumping, aerial stabilization, and landing for traversing martian lava tubes.
