\section{Link Length Optimization}

\section{Link Length Optimization}

\subsection{Motivation}
The robot must have reasonable jumping performance to be suitable for training and demonstrating the future RL control policies. One of the main design aspects that determines the jumping performance is the length of the thigh and shank links. The link lengths impact the robots mass, mass distribution, and acts as constraints for initial jumping poses and spring compression. Which in turn affect how jumps are performed and thus jumping performance. To find the optimal link lengths, a grid search was performed. To do this, a simplified model of the robot was created in Simscape, the details of this are covered in the section \ref{sec:modeling}.


\subsubsection{Pendulum Modeling}

The pendulum used in the motor friction tests consists of an aluminum rod of length \( l_{\text{arm}} = 0.19 \) meters and a ballast mass \( m_{\text{ballast}} = 0.301 \) kg attached at a distance \( r = 0.08 \) meters from the pivot. The total mass of the arm is \( m_{\text{arm}} = 0.034 \) kg. The pendulum is modeled as a rigid body rotating about the motor shaft with a moment of inertia \( I \) given by:

\[
I = \frac{1}{3} m_{\text{arm}} l_{\text{arm}}^2 + m_{\text{ballast}} r^2
\]

The equation of motion for the pendulum, considering only viscous friction, is:

\[
I \ddot{\theta} + b \dot{\theta} + (m_{\text{arm}} \frac{l_{\text{arm}}}{2} + m_{\text{ballast}} r) g \sin(\theta) = 0
\]

where:
\begin{itemize}
    \item \( \theta \) is the angular displacement (positive counterclockwise, zero at vertical down position)
    \item \( \dot{\theta} \) and \( \ddot{\theta} \) are the angular velocity and acceleration, respectively
    \item \( b \) is the viscous damping coefficient
    \item \( g = 9.81 \, \text{m/s}^2 \) is the acceleration due to gravity
\end{itemize}
\subsubsection{Linear Regression Derivation}
Rearranging the equation for linear regression purposes:

\[
    I \ddot{\theta} + (m_{\text{arm}} \frac{l_{\text{arm}}}{2} + m_{\text{ballast}} r) g \sin(\theta) = -b \dot{\theta}
\]

This can be expressed in the form:

\[
Y = X \beta
\]

where:
\begin{itemize}
    \item \( Y = -I \ddot{\theta} - (m_{\text{arm}} \frac{l_{\text{arm}}}{2} + m_{\text{ballast}} r) g \sin(\theta) \),
    \item \( X = \dot{\theta} \),
    \item \( \beta = b \).
\end{itemize}

The angular velocity \( \dot{\theta} \) and acceleration \( \ddot{\theta} \) are computed using centered finite differences:

\[
\dot{\theta}_i = \frac{\theta_{i+1} - \theta_{i-1}}{2\Delta t}
\]

\[
\ddot{\theta}_i = \frac{\theta_{i+1} - 2\theta_i + \theta_{i-1}}{(\Delta t)^2}
\]

where \(\Delta t\) is the time step between measurements.

The linear least squares solution for \( \beta \) is given by:

\[
\beta = (X^T X)^{-1} X^T Y
\]

This yields the viscous damping coefficient \( b \).




\subsection{Initial Pose Calculation}
- want to avoid sliding on the ground
- want to load spring maximally
- want to avoid knee placed under ground



\subsection{Symmetric versus Asymmetric legs}

\subsection{Grid Search}