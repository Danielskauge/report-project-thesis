\section{Introduction}
\label{sec:introduction}

\subsection{Motivation}
\label{sec:motivation}

Traditional wheeled rovers have successfully explored Mars, with six missions to date \cite{mars_rovers_x6}. However, specialized robots like Axel \cite{Axel} and Reachbot \cite{ReachBot} are needed for more challenging terrain. One key target for exploration is Martian and Lunar lava tubes \cite{lavatubes} - hollow caverns formed by ancient lava flows. These tubes interest scientists because they could shelter future missions from radiation and micrometeorites, maintain stable temperatures, and contain subsurface water ice \cite{lavatubes}.

Exploring lava tubes requires navigating rough terrain, sharp rocks, and steep slopes - challenges for wheeled rovers. Wheeled robots are also limited to ground movement and cannot take advantage of the lower gravity on Mars, the Moon, and asteroids. In contrast, jumping quadrupeds can utilize low gravity to potentially jump several meters high \cite{OLYMPUS2}, allowing them to cross obstacles that would stop wheeled rovers.

While quadruped robots have advanced significantly, they still struggle to jump effectively in Earth gravity. Testing hardware and control systems for low-gravity jumping is difficult since we cannot easily simulate low gravity on Earth. The high-speed impacts during jumping and landing also risk damaging expensive hardware. These challenges motivate our main goal: designing a small, lightweight, low-cost jumping quadruped robot. The low weight and cost reduce damage risk during testing and make the platform more accessible to researchers. We focus on achieving long jumps while maintaining the versatility of quadruped robots, such as walking on rough terrain, adjusting body pose, and carrying scientific equipment.

\subsection{Scope}
\label{sec:scope}

This report covers the design phase of a jumping quadruped robot as part of the TTK4550 Engineering Cybernetics Specialization Project at NTNU. While this project focuses on design, the work will continue in a master's thesis that includes building, testing, and developing control algorithms for the robot.
The project scope:
\begin{itemize}
    \item Create a MATLAB / Simulink Simscape \cite{simulink_simscape} simulation environment for design evaluation and optimization of jumping performance
    \item Select an actuation method using motors and/or springs
    \item Choose hardware components like motors and springs
    \item Develop a CAD model for one leg that:
    \begin{itemize}
        \item Fits the chosen motors and springs
        \item Can be manufactured using NTNU's 3D printing and CNC facilities
        \item Withstands jumping forces and impacts
    \end{itemize}
\end{itemize}

\subsection{Related Work}
\label{sec:related_work}

Several researchers have studied robotic jumping for Earth and low-gravity environments. NTNU's Autonomous Robots Lab developed the Olympus robot \cite{OLYMPUS1} \cite{OLYMPUS2}, which uses a spring-assisted 5-bar linkage leg for jumping. The robot weighs around 15 kg and should be able to jump to a height of about 4m in Mars gravity \cite{OLYMPUS2}. Experiments on a 7.9kg bipedal prototype version of the robot has demonstrated jump heights of approximately 1.1m in earth gravity \cite{OLYMPUS1}. 

EPFL's 600g RAVEN robot (Robotic Avian-inspired Vehicle for multiple ENvironments) \cite{RAVEN} uses bird-inspired legs with two degrees of freedom. Similarly to our approach, it uses geared BLDC motors to wind up embedded torsional springs for jumping. By jumping, the 50 cm long RAVEN robot can land on a 26 cm tall obstacle \cite{RAVEN}. 
%RAVEN can jump TODO cm in Earth gravity while also walking and hopping like a bird.

The 15g Grillo robot \cite{GRILLO} demonstrates high-speed jumping, reaching takeoff velocities of 1.5 m/s (30 body lengths per second).